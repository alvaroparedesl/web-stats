\documentclass[]{article}
\usepackage{lmodern}
\usepackage{amssymb,amsmath}
\usepackage{ifxetex,ifluatex}
\usepackage{fixltx2e} % provides \textsubscript
\ifnum 0\ifxetex 1\fi\ifluatex 1\fi=0 % if pdftex
  \usepackage[T1]{fontenc}
  \usepackage[utf8]{inputenc}
\else % if luatex or xelatex
  \ifxetex
    \usepackage{mathspec}
  \else
    \usepackage{fontspec}
  \fi
  \defaultfontfeatures{Ligatures=TeX,Scale=MatchLowercase}
\fi
% use upquote if available, for straight quotes in verbatim environments
\IfFileExists{upquote.sty}{\usepackage{upquote}}{}
% use microtype if available
\IfFileExists{microtype.sty}{%
\usepackage{microtype}
\UseMicrotypeSet[protrusion]{basicmath} % disable protrusion for tt fonts
}{}
\usepackage[margin=1in]{geometry}
\usepackage{hyperref}
\hypersetup{unicode=true,
            pdfborder={0 0 0},
            breaklinks=true}
\urlstyle{same}  % don't use monospace font for urls
\usepackage{natbib}
\bibliographystyle{plainnat}
\usepackage{longtable,booktabs}
\usepackage{graphicx,grffile}
\makeatletter
\def\maxwidth{\ifdim\Gin@nat@width>\linewidth\linewidth\else\Gin@nat@width\fi}
\def\maxheight{\ifdim\Gin@nat@height>\textheight\textheight\else\Gin@nat@height\fi}
\makeatother
% Scale images if necessary, so that they will not overflow the page
% margins by default, and it is still possible to overwrite the defaults
% using explicit options in \includegraphics[width, height, ...]{}
\setkeys{Gin}{width=\maxwidth,height=\maxheight,keepaspectratio}
\IfFileExists{parskip.sty}{%
\usepackage{parskip}
}{% else
\setlength{\parindent}{0pt}
\setlength{\parskip}{6pt plus 2pt minus 1pt}
}
\setlength{\emergencystretch}{3em}  % prevent overfull lines
\providecommand{\tightlist}{%
  \setlength{\itemsep}{0pt}\setlength{\parskip}{0pt}}
\setcounter{secnumdepth}{5}
% Redefines (sub)paragraphs to behave more like sections
\ifx\paragraph\undefined\else
\let\oldparagraph\paragraph
\renewcommand{\paragraph}[1]{\oldparagraph{#1}\mbox{}}
\fi
\ifx\subparagraph\undefined\else
\let\oldsubparagraph\subparagraph
\renewcommand{\subparagraph}[1]{\oldsubparagraph{#1}\mbox{}}
\fi

%%% Use protect on footnotes to avoid problems with footnotes in titles
\let\rmarkdownfootnote\footnote%
\def\footnote{\protect\rmarkdownfootnote}

%%% Change title format to be more compact
\usepackage{titling}

% Create subtitle command for use in maketitle
\newcommand{\subtitle}[1]{
  \posttitle{
    \begin{center}\large#1\end{center}
    }
}

\setlength{\droptitle}{-2em}
  \title{}
  \pretitle{\vspace{\droptitle}}
  \posttitle{}
  \author{}
  \preauthor{}\postauthor{}
  \date{}
  \predate{}\postdate{}

\usepackage{booktabs}

\begin{document}

{
\setcounter{tocdepth}{2}
\tableofcontents
}
\section{Probabilidad}\label{prob-2}

Intuitivas en parte, no tanto en varias. Todo es posible, pero a la vez
imposible. Sí, de eso se tratan las probabilidades.

La probabilidad no se puede dar por definición, se deriva de axiomas
(que veremos en \ref{tconjuntos-2}) en base a una función de
probabilidad. Cualquier cosa que cumpla con los estos 3 axiomas, es una
probabilidad. La definición filosófica, la dejamos para otro momento,
por ahora nos quedaremos con la definición matemática.

\subsection{Conceptos importantes}\label{cimportantes-2}

\subsubsection{Muestras significativas}\label{msignificativas-2}

Tal y como nos referimos en \ref{cbasicos-1}, una muestra no puede ser
calificada de \emph{``significativa''}. Como la mayoría de las veces
desconocemos las características que nos interesan de la población, es
imposible saber si dicha muestra es o no es significativa. Por lo tanto
el concepto es absurdo. Lo que sí es necesario de una muestra, es que su
origen debe ser probabilístico, lo que se traduce en que debe ser
posible asignarle una probabilidad de ser seleccionada (probabilidad de
obtener dicha muestra).

¿De qué tamaño debe ser la muestra? Depende de la variabilidad de la
población. Pero diantres, si la población es desconocida, ¿cómo sabemos
todas esas cosas? Bueno, bienvenido a los supuestos y simplificaciones,
que de esos tendremos por montón. Lo bueno, es que seremos capaces de
cuantificarlos.

\subsubsection{Sesgo}\label{sesgo}

La cualidad de sesgado o insesgado es relativa al estimador, no a la
muestra.

\subsection{Teoría de conjuntos}\label{tconjuntos-2}

\begin{itemize}
\tightlist
\item
  Contenido: \(A \subset B\)
\item
  Igualdad: \(A = B \Leftrightarrow A \subset B\) y \(B \subset A\)
\item
  Unión: \(A \cup B\)
\item
  Intersección: \(A \cap B\)
\item
  Complemento: \(A^c\)
\end{itemize}

\subsubsection{Teoremas}\label{teoremas}

\begin{enumerate}
\def\labelenumi{\arabic{enumi}.}
\tightlist
\item
  Conmutatividad: \(A \cup B = B \cup A\) y \(A \cap B = B \cap A\)
\item
  Asociatividad: \(A \cup (B \cup C) = (A \cup B) \cup C\) y
  \(A \cap (B \cap C) = (A \cap B) \cap C\)
\item
  Distributiva: \(A \cup (B \cap C) = (A \cup B) \cap (A \cup C)\) y
  \(A \cap (B \cup C) = (A \cap B) \cup (A \cap C)\)
\item
  Complemento: \((A \cup B)^c = (A^c \cap B^c)\) y
  \((A \cap B)^c = (A^c \cup B^c)\)
\end{enumerate}

\subsection{Teoría de probabilidad}\label{teoria-de-probabilidad}

Debemos definir:

\begin{enumerate}
\def\labelenumi{\arabic{enumi}.}
\tightlist
\item
  Espacio muestral: conjunto de todos los resultados posibles
  (\(\Omega\))
\item
  Familia de eventos \(\Im\) (\(\sigma\)-álgebra o \emph{Borel field}):
  subconjunto de \(\Omega\) a los cuales les asigno una probabilidad
  (\(P\)). Cobra importancia e interés, cuando \(\Omega\) no es
  numerable o es infinito. En el fondo es una partición de \(\Omega\)
  que puede contener la misma cantidad de elementos (partición fina,
  caso finito o numerable) o menos elementos (partición gruesa, para
  casos no numerables o infinitos).
\item
  Probabilidad \(P\): definición de función de probabilidad
\end{enumerate}

\subsubsection{\texorpdfstring{Características de
\(\Im\)}{Características de \textbackslash{}Im}}\label{caracteristicas-de-im}

Para que \(\Im\) sea considerada una \(\sigma\)-álgebra, debe tener las
siguientes propiedades:

\begin{enumerate}
\def\labelenumi{\arabic{enumi}.}
\tightlist
\item
  \(\emptyset \in \Im\)
\item
  Si \(A \in \Im\), entonces \(A^c \in \Im\)
\item
  Si
  \(A_1, A_2, ... \in \Im \Rightarrow \bigcup_{i=1}^\infty A_i \in \Im\)
\end{enumerate}

Cualquier cosa que cumpla estas propiedades, es una \(\sigma\)-álgebra.
Una \(\sigma\)-álgebra contiene \(2^n\) conjuntos, donde \(n\) es la
cantidad de elementos contenidos.

\subsubsection{\texorpdfstring{Características de una función de
probabilidad \(P\)
(\(\Omega, \Im, P\))}{Características de una función de probabilidad P (\textbackslash{}Omega, \textbackslash{}Im, P)}}\label{caracteristicas-de-una-funcion-de-probabilidad-p-omega-im-p}

Definida como: \(P: \Im \rightarrow \mathbb{R} \\\)
\(A \rightarrow P(A)\)

Y dado un espacio muestral \(\Omega\) y una \(\Im\) asociada a dicho
espacio muestral, \(P\) es una probabilidad de función (con dominio en
\(\Im\)) si cumple los siguientes 3 axiomas de probabilidad:

\begin{enumerate}
\def\labelenumi{\arabic{enumi}.}
\tightlist
\item
  \(P(A) \ge 0, \forall A \in \Im\)
\item
  \(P(\Omega) = 1\)
\item
  Si \(A_1, A_2, ... \in \Im\) y son pares disjuntos, entonces
  \(P(\bigcup_{i=1}^\infty A_i) = \sum_{i=1}^\infty P(A_i)\)
\end{enumerate}

Si \(P\) es una función de probabilidad, con \(A \in \Im\), entonces se
cumple el siguiente teorema:

\begin{enumerate}
\def\labelenumi{\arabic{enumi}.}
\tightlist
\item
  \(P(\emptyset) = 0\)
\item
  \(P(A) \leq 1\)
\item
  \(P(A^c) = 1- P(A)\)
\end{enumerate}

Si \(P\) es una función de probabilidad, con \(A,B \in \Im\), entonces
se cumple el siguiente teorema:

\begin{enumerate}
\def\labelenumi{\arabic{enumi}.}
\tightlist
\item
  \(P(B \cap A^c) = P(B) - P(A \cap B)\)
\item
  \(P(A \cup B) = P(A) + P(B) - P(A \cap B)\)
\item
  Si \(A \subset B \Rightarrow P(A) \leq P(B)\)
\end{enumerate}

En el segundo caso, cuando los eventos son disjuntos,
\(A \cap B = \emptyset\)

\subsubsection{Probabilidades en poblaciones finitas: permutaciones y
combinaciones}\label{probabilidades-en-poblaciones-finitas-permutaciones-y-combinaciones}

En el caso finito y/o contable (como los números enteros), tenemos:

\begin{itemize}
\tightlist
\item
  Permutaciones: Arreglo ordenado de objetos, con o sin
  reposición/reemplazo. Es decir, el orden sí importa; elegir entre dos
  letras \(A\) y \(B\), son elementos distintos \(AB\) y \(BA\).
\item
  Combinaciones: Arreglo no ordenado de objetos, con o sin
  reposición/reemplazo. El orden no importa; elegir entre dos letras
  \(A\) y \(B\), son el mismo elemento \(AB\) y \(BA\).
\end{itemize}

En el caso equipobrable (cada evento tiene probabilida de ocurrencia
\(1/n\)), se tiene lo siguiente, siendo \(r\) el tamaño de la muestra y
\(n\) un conjunto dado:

\label{tab:permutaciones-combinaciones-equi-2}Cuadro resumen para contar
eventos ordenados o no ordenados, con o sin reposición.

Con reemplazo

Sin reemplazo

Ordenado (Permutaciones)

\(n^r\)

\(\frac{n!}{(n-r)!}\)

No ordenado (Combinaciones)

\(\frac{(n + r - 1)!}{r!(n-r)!} = \binom{n + r - 1}{r}\)

\(\frac{n!}{r!(n-r)!} = \binom{n}{r}\)

Como se puede apreciar en la tabla
\ref{tab:permutaciones-combinaciones-equi-2}, las combinaciones tienen
una notación especial denominada por \(\binom{}{}\), que es una
abreviatura para dicha formulación. Si pensamos las cuatro expresiones
que aparecen, podemos deducir que ante un mismo evento, las
permutaciones (donde el orden importa) entregarán un número mayor de
eventos posibles, con respecto a las combinaciones. Por otro lado,
cuando hay reposición, el número de eventos que se pueden generar
también es superior a cuando no hay reposición. Son ideas un poco obvias
e intuitivas, pero que vale la pena resaltar.

Veamos un ejemplo de cada caso y como se llega a las formulas resumen de
la tabla \ref{tab:permutaciones-combinaciones-equi-2}, partiendo de la
pregunta:

\emph{Si tengo 6 bolas de colores ¿de cuántas maneras puedo tomar 3
bolas?}

\begin{enumerate}
\def\labelenumi{\arabic{enumi}.}
\item
  Permutación con reposición: Si antes de tomar la siguiente bola,
  devuelvo la que saqué previamente. La primera vez, puedo sacar 6
  bolas, la segunda vez también puedo sacar 6 bolas pues he repuesto la
  que saqué anteriormente y la tercera vez, también puedo tomar 6 bolas.
  Esto es equivalente a \(6 * 6 * 6 = 6^3 = 216\). Es decir, puedo
  hacerlo de 216 formas posibles.
\item
  Permutación sin reposición: Si repito el ejercicio el anterior, sin
  devolver las bolas que voy sacando, obtengo el resultado de
  multiplicar \(6 * 5 * 4 = 120\). Esta operación, la podemos anotar
  como
  \(\frac{n!}{(n-r)!} = \frac{12!}{(12-3)!} = \frac{6!}{3!} = \frac{6*5*4*3!}{3!}\).
  Cancelando \(3!\), obtenemos la multiplicación deseada. Notar que de
  forma intuitiva, no podemos obtener un resultado mayor al experimento
  con reposición, pues cada vez vamos teniendo menos opciones.
\item
  Combinación sin reposición: Similar al de permutación con reposición,
  pero ahora hay que descontar las veces en que la elección tenía los
  mismos elemntos, pero ordenados de forma diferente. Esto sería
  \(\binom{n}{r} = \frac{n!}{r!(n-r)!} = \frac{6!}{3!(6-3)!} = \frac{6*5*4}{3!} = 20\).
  En este caso, el \(3!\) está contando de cuantas maneras se pueden
  elegir esas 3 bolas ordenadamente sin reposición y como está en el
  denominador, estamos descontando dichas combinaciones del espacio de
  solución. Nuevamente es bueno notar que el resultado esperado, también
  es un número menor al de una permutación sin reposición (y en
  realidad, debiera ser siempre el menor de los cuatro casos), pues el
  orden esta vez no nos interesa, y con ello caen automáticamente el
  número de casos.
\item
  Combinación con reposición: Este no es muy similar a nada, y es un
  poco complejo de explicar. Es mejor ver primero la aplicación, que
  sería
  \(\binom{n + r - 1}{r} = \frac{(n + r - 1)!}{r!((n + r - 1) - r)!} = \frac{(6 + 3 - 1)!}{3!((6 + 3 - 1) - 3)!} = \frac{8*7*6*5!}{3! 5!} = 56\).
  El término \(r\) queda eliminado en el denominador, con lo que en el
  fondo queda \(\frac{(n + r - 1)!}{r!(n - 1)!}\). Es de esperar que al
  ser con reposición, se obtenga un número superior a sin reposición, y
  obviamente, por debajo de al menos el caso de permutación con
  reposición.
\end{enumerate}

\subsubsection{Probabilidad condicional e
independencia}\label{probabilidad-condicional-e-independencia}

Cuando se quiere conocer la probabilidad de que se produzca un evento
\(A\), dado que ocurrió un evento \(B\), hablamos de probabilidad
condicional. Esto está expresado en la ecuación
\eqref{eq:probabilidad-condicional}, siempre que se cumpla \(P(B) > 0\).

\begin{equation}
P(A|B) = \frac{P(A \cap B)}{P(B)}
\label{eq:probabilidad-condicional}
\end{equation}

Notar que la ecuación \eqref{eq:probabilidad-condicional} es equivalente a
la expresion \eqref{eq:probabilidad-condicional-ext}, gracias a las
propiedades de simetría.

\begin{equation}
P(A \cap B) = P(A|B) * P(B) = P(B|A) * P(A)
\label{eq:probabilidad-condicional-ext}
\end{equation}

Y basados en la ecuación \eqref{eq:probabilidad-condicional-ext}, podemos
obtener también:

\begin{equation}
P(A|B) = P(B|A) \frac{P(A)}{P(B)}
\label{eq:probabilidad-condicional-ext2}
\end{equation}

Si la probabilidad de que ocurra \(P(A|B)\) es la misma que \(P(A)\),
entonces el evento \(A\) es independiente de \(B\) y podemos resolver
como se muestra en la ecuación
\eqref{eq:eventos-independientes-desarrollo}.

\begin{align}
P(A|B) &= \frac{P(B|A) * P(A)}{P(B)} \\
       &= P(B|A) \frac{P(A)} {P(B)} \\
       &= P(B) \frac{P(A)} {P(B)} \\
       &= P(A)
\label{eq:eventos-independientes-desarrollo}
\end{align}

La expresión \eqref{eq:eventos-independientes-desarrollo} se deriva
entonces, en la expresión \eqref{eq:eventos-independientes}.

\begin{equation}
P(A|B) * P(B) = P(A \cap B) = P(A) * P(B)
\label{eq:eventos-independientes}
\end{equation}

Importante notar, que los eventos disjuntos son completamente distintos
a la independencia. La independencia corresponde a la función de
probabilidad utilizada \(P\), mientras ser disjuntos es una propiedad de
los eventos de \(\Im\). Así, para la misma \(\Im\), pueden existir
diferentes funciones de probabilidad, en donde algunas de esas
functiones sean independientes para \(P(A)\) y \(P(B)\), mientras que en
otro modelo no lo sean (por ejemplo, el evento de comprar o no un yogurt
la probabilidad es dependiente si se utiliza sobre el sabor, pero
independiente sobre el código de barras).

Por la misma razón, la única forma en que \(P(A)\) y \(P(B)\) sean
disjuntos e independientes, es que alguno de los dos sea 0 dado que es
la única forma de cumplir con \(P(A) + P(B) = P(A) * P(B)\).

Si \(A\) y \(B\) son eventos independientes, entonces se cumple el
siguiente teorema, que indica que también son independientes:

\begin{enumerate}
\def\labelenumi{\arabic{enumi}.}
\tightlist
\item
  \(A\) y \(B^c\)
\item
  \(A^c\) y \(B\)
\item
  \(A^c\) y \(B^c\)
\end{enumerate}

Los eventos \(A_1, ..., A_n\) son independientes dos a dos si se
verifica que
\(P(A_i \cap B_j) = P(A_i) * P(B_j), \forall \text{ } i \neq j\) con
\(i,j = 1, ..., n\).

Los eventos \(A_1, ..., A_n\) son mutuamente independientes si son
independientes entre sí, en cualquier combinación y número (dos a dos,
dos a tres, tres a tres, etc).

\paragraph{Probabilidad total}\label{probabilidad-total}

\begin{equation}
\sum_{i=1}^nP(A_i)P(B|A_i)
\label{eq:probabilidad-total}
\end{equation}

\paragraph{Bayes}\label{bayes}

\begin{equation}
P(A_k|B) = \frac{P(A_k)P(B|A_k)}{\sum_{i=1}^nP(A_i)P(B|A_i)}, k=1,...,n
\label{eq:teorema-bayes}
\end{equation}

\subsection{Variable aleatoria}\label{variable-aleatoria}

Es una función medible \(X\) que va de \(\Omega\) a \(\mathbb{R}\).
\(X\) induce una partición en \(\Omega\) y queremos que \(X\) esté en
\(\Im\); por otro lado \(X\) debe ser igual o más gruesa que \(\Im\),
pero no menor, o de lo contrario no habrá una probabilidad \(P\), dado
que \(P\) está definida sobre los eventos y al estar los eventos en
\(\Im\), una partición menor implica que el evento no existe en \(\Im\).

Una variable aleatoria es una función, pero no cualquier función es una
variable aleatoria. En dicha función, lo que va cambiando es el
recorrido, pero no el dominio (que es \(\Omega\)).

Definida como: \(X(x): \Omega \rightarrow \mathbb{R} \\\)
\(\Im \leftarrow X^{-1}(x)\)

Y formalmente está definida en la ecuación \eqref{eq:variable-aleatoria} y
también en la ecuación \eqref{eq:variable-aleatoria2}. Ambas son
equivalentes.

\begin{equation}
\{\omega \in \Omega:X(\omega) \leq x\} \in \Im, \forall \text{ } x \in \mathbb{R}
\label{eq:variable-aleatoria}
\end{equation}

\begin{equation}
\{\omega \in \Omega:X(\omega) \in B\} \in \Im, \text{donde } B \text{ es cualquier conjunto Boreliano}
\label{eq:variable-aleatoria2}
\end{equation}

Esto me permite calcular \(P\) para cualquier intervalo que se quiera.

\subsubsection{Función de distribución de
probabilidad}\label{funcion-de-distribucion-de-probabilidad}

Dado un espacio de probabilidad \((\Omega, \Im, P)\), se considera una
\textbf{función de distribución} de la variable aleatoria \(X\)
(\(F_X(x)\)) para todos los valores \(x\) reales, mediante la ecuación
\eqref{eq:funcion-distribucion}. En inglés, se denomina \textbf{Cumulative
Distribution Function}, o CDF.

\begin{equation}
F_X(x) = P_X(X \leq x) \equiv P(\{\omega \in \Omega: X(\omega) \leq x \}) 
\label{eq:funcion-distribucion}
\end{equation}

Por otro lado, si se considera para el mismo espacio de probabilidad
\((\Omega, \Im, P)\) la variable aleatoria \(X = X(\omega) \in \Omega\),
entonces esa variable aleatoria es \textbf{función de probabilidad} si
cumple la ecuación \eqref{eq:funcion-probabilidad}, para todos los
conjuntos Borelianos \(B\).

\begin{equation}
P_X(B) = P_X(X \in B) = P(\{\omega : X(\omega) \in B\})
\label{eq:funcion-probabilidad}
\end{equation}

Finalmente, una \textbf{distribución de probabilidad} de una variable
aleatoria \(X\) es la función de distribución \(F_X(x)\) (ecuación
\eqref{eq:funcion-distribucion}) o la función de probabilidad \(P_X(B)\)
(ecuación \eqref{eq:funcion-probabilidad}).

Algunas propiedades importantes:

\begin{enumerate}
\def\labelenumi{\arabic{enumi}.}
\tightlist
\item
  \(\forall \in \mathbb{R} \Rightarrow 0 \leq F_X(x) \leq 1\)
\item
  \(a, b \in \mathbb{R} \Rightarrow P_X(a < X \leq b) = F_X(b) - F_X(a)\)
\item
  \(\forall \text{ } a, b \in \mathbb{R} \Rightarrow F_X(a) \leq F_X(b)\)
  (no decreciente en la recta real)
\item
  \(\lim_{x \rightarrow + \infty} \Rightarrow F_X(x) = 1\) y
  \(\lim_{x \rightarrow - \infty} \Rightarrow F_X(x) = 0\)
\item
  \(\forall \text{ } x_0, \lim_{x \rightarrow x_0^+} = F_X(x_0)\)
  (continuidad por la derecha, es decir cuando va de \(+\infty\) a
  \(-\infty\))
\item
  \(F_X(x)\) tiene una cantidad finita o numerable de puntos
  discontinuos
\end{enumerate}

Una variable aleatoria es discreta, si \(F_X(x)\) es una función
discreta de \(x\).

Una variable aleatoria es continua, si \(F_X(x)\) es una función
continua de \(x\).

Las variables aleatorias \(X\) e \(Y\) son idénticamente distribuidas,
si \(F_X(x) = F_Y(y), \text{ } \forall x\). Es importante tener claro
que eso no quiere decir que \(X=Y\) o que \(X \neq Y\).

La función que indica la probabilidad de ocurrencia de un
\emph{``punto''} en particular, se denomina \textbf{Función de
Probabilidad de Masa} (\textbf{Probability Mass Function}, PMF) y
\textbf{Función de Densidad de Probabilidad} (\textbf{Probability
Density Function}, PDF), para el caso discreto y continuo
respectivamente. En el caso continuo, el punto tiene probabilidad 0, por
lo que en realidad solo está estipulado para entenderlo de manera
conceptual.

La PMF se escribe como \(f_X(x)\), para diferenciarla de la PDF
\(F_X(x)\). Ambas están definidas en la ecuación \eqref{eq:PMF-def} y
\eqref{eq:PDF-def}.

\begin{equation}
PMF = f_X(x) = P(X=x), \text{} \forall x
\label{eq:PMF-def}
\end{equation}

\begin{equation}
PDF = F_X(x) = \int_{-\infty}^{x} f_X(t) dt, \text{} \forall x
\label{eq:PDF-def}
\end{equation}

\subsubsection{Variable aleatoria
discreta}\label{variable-aleatoria-discreta}

Una variable aleatoria tiene distribución discreta si existe un conjunto
\(B\) finito o numerable de la recta real, tal que \(P(X \in B) = 1\)

Una \textbf{función de masa de probabilidad} está dada por \(P_X(X=x)\)
para todo \(x\). La variable aleatoria tiene una distribución degenerada
si existe un número \(c\) y \(P(X=c) = 1\).

\subsubsection{Variable aleatoria
continua}\label{variable-aleatoria-continua}

Me niego


\end{document}
